%%%%%%%%%%%%%%%%%%%%%%%%%%%%%%%%%%%%%%%%%%%%%%%%%%%%%%%%%%%%%%%
%
% Welcome to Overleaf --- just edit your LaTeX on the left,
% and we'll compile it for you on the right. If you open the
% 'Share' menu, you can invite other users to edit at the same
% time. See www.overleaf.com/learn for more info. Enjoy!
%
%%%%%%%%%%%%%%%%%%%%%%%%%%%%%%%%%%%%%%%%%%%%%%%%%%%%%%%%%%%%%%%
\documentclass{article}
\usepackage[
backend=biber,
style=ieee,
]{biblatex}
\title{Annotated Bibliography}

\addbibresource{sample.bib} %Imports bibliography file

\begin{document}

\maketitle

All citations are in IEEE Style, as I want some foray into writing academic
papers in the computer science field.

% % A brief annotation (150-200 words) which does all four of the following:%
% Identifies the purpose/s of the source (What change do you think the author
% hoped would happen when people read this source?)% Explains who you think
% the intended audience of the source is, and why you think this% Explains some
% key strengths of the source and how it can be useful as you investigate your
% topic% Explains possible limitations or weaknesses you see in the source%
% (e.g., credibility, scope of information, author bias, etc.) Keep in mind%
% that you might have to do a little additional research to get some other%
% perspectives on your source to evaluate its reliability (e.g., book reviews,%
% author bios, original publication dates, etc.)

\section{Book}

Citation: \cite{stokes2021}:
\\
Link : \url{https://ucsc.primo.exlibrisgroup.com/permalink/01CDL_SCR_INST/15r5l0d/alma991025892758804876}
\\

The purpose of the book is to educate those looking into copyright law for
art and the various rules that apply to different types of art, even those
of non-conventional mediums. As for the intention of the author writing this
book, I believe he desired to provide a compilation of guidelines regarding
various areas that copyright touches on regarding art and derivative works. I'm
rationalizing his purpose as the main consumers of his book are those who work
in art law, IP law, law students, art house managers, and others in the field,
who regard his book as a critical source of truth regarding common copyright
misunderstandings and education. I intend to use this source to understand
the basics of copyright law and gain some insight on how copyright law has
applied to derivative works in the past. In essence, I plan to use this source
to develop the foundation of copyright that my essay will build upon. Some
weaknesses with this source are that it was published in 2001, almost 24
years ago, so I doubt that it'll include anything of value regarding AI.

\section{Newspaper Article}

Citation: \cite{thomas2024}
\\
Link : \url{https://ucsc.primo.exlibrisgroup.com/permalink/01CD_SCR_INST/ojisf2/cdi_proquest_newspapers_2937453324}
\\

The purpose of this article is to report the recent developments regarding
the failure of the UK to create a regulatory body to govern AI training
and usage of copyrighted material. This source is useful for my research
because it goes into detail about the struggles present regulatory bodies
have regarding the establishment of an AI regulatory body. The failures are a result
of lack of understanding / judicial precedent. The audience for
this article would be the daily readers of "The Financial Times" and those
interested in the developments of AI regulatory bodies. Some weaknesses with
this article are that it's fairly short, so it might not go as deep as I
would like, which might require me to investigate other sources. I definetly want to
investigate more articles, especially more in the programming space as I'm leaning
for the core of my paper being focused on Code outputted by AI and the legality behind that.

\section{Paper}

Citation: \cite{ma2024}
\\
Link : \url{https://ucsc.primo.exlibrisgroup.com/permalink/01CD_SCR_INST/ojisf2/cdi_ieee_primary_10440501}
\\

The purpose of this academic paper is to share their academic findings on
new techniques to identify the source of code, i.e., whether it's from an
AI or a human, and even possibly the origin of that code. This is extremely
relevant to my project because the fact that active research is being done
on tracing the sources of derivative AI works could, in the near future,
make for a comprehensive case on the distinction between fair use and
plagiarism. I'm quite excited to read the paper, but one downside I can
think of is that it's fairly new, so it might not be fully peer-reviewed,
but it should be interesting nonetheless. The paper also talks about some
open source tools they created that can identify the origins of some
code and will tell you if it comes from any of the code repositories online
that they have indexed. Im quite excited to give this a try, I might even be
able to show some results in my paper!

\section{Video}

Citation: \cite{bloomberglaw2023}
\\
Link : \url{https://www.youtube.com/watch?v=bRqwTP2eKJY}
\\

The purpose of this video is to inform the general public about recent
developments in artificial intelligence and the capability for copyright
law to stop them, as it currently stands. This source is going to be useful
because its focus is purely on information regarding the topic at hand and
comes from a reliable source, Bloomberg Law. The audience for this would
be the general public on YouTube who happen to have this video recommended,
and maybe some astute followers of Bloomberg Law. Some downsides to this are
that it is a relatively short video, so I'm a bit concerned about its depth,
but other than that, it's a credible source with a lot of relevance. I'm
probably going to end up looking for some more videos that
go deeper into depth in certain areas, like any existing cases regarding AI and copyright,
and how those went.


\medskip

\printbibliography
\end{document}
