%%%%%%%%%%%%%%%%%%%%%%%%%%%%%%%%%%%%%%%%%%%%%%%%%%%%%%%%%%%%%%%
%
% Welcome to Overleaf --- just edit your LaTeX on the left,
% and we'll compile it for you on the right. If you open the
% 'Share' menu, you can invite other users to edit at the same
% time. See www.overleaf.com/learn for more info. Enjoy!
%
%%%%%%%%%%%%%%%%%%%%%%%%%%%%%%%%%%%%%%%%%%%%%%%%%%%%%%%%%%%%%%%
\documentclass{article}
\usepackage[
backend=biber,
style=ieee,
]{biblatex}
\title{Annotated Bibliography}

\addbibresource{sample.bib} %Imports bibliography file

\begin{document}

\maketitle

All citations are in IEEE Style, as I want some foray into writing academic papers in
the computer science field.


% A brief annotation (150-200 words) which does all four of the following:
% Identifies the purpose/s of the source (What change do you think the author hoped would happen when people read this source?) 
% Explains who you think the intended audience of the source is, and why you think this
% Explains some key strengths of the source and how it can be useful as you investigate your topic
% Explains possible limitations or weaknesses you see in the source
% (eg. credibility, scope of information, author bias, etc.) Keep in mind
% that you might have to do a little additional research to get some other
% perspectives on your source to evaluate its reliability (eg. book reviews,
% author bios, original publication dates etc.)

\section{Book}

Citation: \cite{stokes2021}
Link: \url{https://ucsc.primo.exlibrisgroup.com/permalink/01CDL_SCR_INST/15r5l0d/alma991025892758804876}

The purpose of the book is to educate those looking into copyright law
for art and the various rules that apply to different types of art, even those of non-conventional medium.
As for the intention of the author writing this book, I believe he desired to
provide a compilation of guidelines regarding various different areas that copyright touches on
regarding art and derivative works. I'm rationalizing his purpose as the main consumers
of his book are those who work in art law, ip law, law students, art house managers,
and others in the field, who regard his book as a critical source of truth regarding
common copyright misunderstandings and education. I intend to use this source
to understand the basics of copyright law and gain some insight on how copyright law
has applied to derivative works in the past. In esscence, I plan to use this source to develop
the foundation of copyright that my essay will build upon. Some weaknesses with this source
is that it was published in 2001, almost 24 years ago, so I doubt that it'll include anything of
value regarding AI.







\section{Newspaper}

Citation: \cite{thomas2024}
Link: \url{https://ucsc.primo.exlibrisgroup.com/permalink/01CD_SCR_INST/ojisf2/cdi_proquest_newspapers_2937453324}

\section{Paper}

Citation: \cite{ma2024}
Link: \url{https://ucsc.primo.exlibrisgroup.com/permalink/01CD_SCR_INST/ojisf2/cdi_ieee_primary_10440501}

\section{Video}

Citation: \cite{bloomberglaw2023}
Link: https://www.youtube.com/watch?v=bRqwTP2eKJY

\medskip

\printbibliography
\end{document}
