%% Lab Report for EEET2493_labreport_template.tex
%% V1.0
%% 2019/01/16
%% This is the template for a Lab report following an IEEE paper. Modified by Francisco Tovar after Michael Sheel original document.


%% This is a skeleton file demonstrating the use of IEEEtran.cls
%% (requires IEEEtran.cls version 1.8b or later) with an IEEE
%% journal paper.
%%
%% Support sites:
%% http://www.michaelshell.org/tex/ieeetran/
%% http://www.ctan.org/pkg/ieeetran
%% and
%% http://www.ieee.org/

%%*************************************************************************
%% Legal Notice:
%% This code is offered as-is without any warranty either expressed or
%% implied; without even the implied warranty of MERCHANTABILITY or
%% FITNESS FOR A PARTICULAR PURPOSE! 
%% User assumes all risk.
%% In no event shall the IEEE or any contributor to this code be liable for
%% any damages or losses, including, but not limited to, incidental,
%% consequential, or any other damages, resulting from the use or misuse
%% of any information contained here.
%%
%% All comments are the opinions of their respective authors and are not
%% necessarily endorsed by the IEEE.
%%
%% This work is distributed under the LaTeX Project Public License (LPPL)
%% ( http://www.latex-project.org/ ) version 1.3, and may be freely used,
%% distributed and modified. A copy of the LPPL, version 1.3, is included
%% in the base LaTeX documentation of all distributions of LaTeX released
%% 2003/12/01 or later.
%% Retain all contribution notices and credits.
%% ** Modified files should be clearly indicated as such, including  **
%% ** renaming them and changing author support contact information. **
%%*************************************************************************

\documentclass[journal]{IEEEtran}

% *** CITATION PACKAGES ***
\usepackage[style=ieee]{biblatex} 
\bibliography{main.bib}    %your file created using JabRef

% *** MATH PACKAGES ***
\usepackage{amsmath}

% *** PDF, URL AND HYPERLINK PACKAGES ***
\usepackage{url}
% correct bad hyphenation here
\hyphenation{op-tical net-works semi-conduc-tor}
\usepackage{graphicx}  %needed to include png, eps figures
\usepackage{float}  % used to fix location of images i.e.\begin{figure}[H]

\begin{document}

% paper title
\title{Analyzing the Effectiveness of Copyright Law on AI Output in Software
Engineering and Future Developments}

% author names 
\author{Surendra Jammishetti, sjammish@ucsc.edu}% <-this % stops a space
        
% The report headers
\markboth{Writing 2 Final Argumentative Essay, February 2025}%do not delete next lines
{Shell \MakeLowercase{\textit{et al.}}: Bare Demo of IEEEtran.cls for IEEE Journals}

% make the title area
\maketitle

% As a general rule, do not put math, special symbols or citations
% in the abstract or keywords.
\begin{abstract}
TODO: need to do this later
\end{abstract}

\begin{IEEEkeywords}
AI, Software Engineering, Copyright
\end{IEEEkeywords}

\section{Introduction}
% want to talk about the problem space
% rise of artificial intelligence
% tools like github copilot, cursor AI editor
% widespread usage by college students,
% talk about github marketing code efficiency 
% just state thesis
%
% 

\IEEEPARstart{T}{he} emergence of AI tools has, no doubt, changed many aspects of
how work is approached in many professions, but the greatest impact has been on
software engineering. New tools like Github Copilot, an AI assisted code generation
tool thats usable in many modern code editors today, boasts a 2x speedup of feature
completions and a large boost in productivity for its users \cite{github}. However
accurate those claims may be, theres no doubt that these tools are being used
widely by the developer population, and they arent going away any time soon.


However the code generated by these tools must come from somewhere. Many models
source their training data from the public internet, but not everything in public access
allows for commercial / derivative use. Such models ignore this step,
jumping straight towards using this data with any disregard for copyright protections in place.
While copyright has been slow in the past, there is no doubt the legal systems around the world
will regulate and form precedent for these utilites; Inevitably placing their users
under risk.
% WARN: not sure this is the cleanest way to segue into my theses 

\textbf {
  For those concerned with the legality of the software they produce, the usage of
AI-generated code should be avoided as their trustworthiness is dubious and
they will become succeptible to copyright law in the near future.
}

\section{Background}

To help understand the context that this paper resides in, below are
lay-person explanations of AI training, Software Licensure, AI tools, and
more.

\subsection{Data Acquisition}

The way AI development labs gather data for their models is
mainly through web scrapers. Web scrapers are programs that access a
immense number of websites, downloading and orgranizing
the all the data they find \cite{Miquido}. The gathered data
is then used to train whichever model they are interested in. 

The modern way for a website to prevent a webscraper from
downloading its contents is a file called a `robots.txt`,
that any honest web scraper would look for first, and obey
the rules inside \cite{robots}. The problem with this "trust me" approach
is that a dishonest, or even badly programmed, web scraper
can come along and look past this file and do whatever they please\cite{robots}.

% TODO: could talk about how copilot has access to github, largest source code repo


\subsection{Software Licenses}

% -An aside / short paragraph about the overlap between copyright and code.
% Talk about Google vs. Oracle and how modern software practices software
% through the idea of licenses. It's really important that this point
% gets understood by the reader, so add examples / pictures from GitHub?
%   - The article by Harvard Law Review going over the Google v. Oracle
%     case should give enough information to summarize the technical details
%     for the audience so they have a basic understanding. I can cite one
%     of the most widely used licenses and show how they apply to the code
%     in repositories, i.e., the MIT license.


While it may come as a suprise to the average person,
even code nowadays has some kind of protection against
copying / un-permitted usage. The mechanism for this
is called a "Software License"\cite{Blackduck}.
The code that these licenses protect is called "source code".
There exist many kinds of licenses, mainly meant for
open use towards the public, but there are a two that are crucial to understand.

\begin{IEEEitemize}
\item Copy Left 
\\
  A "Copy Left" license allows the general public to view and do whatever they
  would like with the source code, but they enforce that any derivative must
  also use the same license. In effect this is a strictly anti-commercial license
  as its quite hard for a buisness to use Copy Left Licensed software as
  anything they produce using that code must have the same
  Copy Left license, and as a result live in the public domain.
  \cite{Blackduck}

\item Proprietary 
\\
A "Proprietary" license are the most restrictive, as they prevent any viewer / user from
copying, modifying, redistrbuting, etc \cite{Blackduck}. They are the defaco type of license for any
commercial code, as they protect code the best, and are legally viable \cite{harvard}. 
    
\end{IEEEitemize}



\subsection{AI-Powered Tools}

The most popular AI tool for developers is Github Copilot, which
helps developers as they write code. Its most important capability, in
the context of this work, is its power to automatically generate code when
you ask. The same capability exists for Claude, OpenAI's models, and others,
where they have the capacity to take a description of what the
user wants, and can write code to perform that task \cite{s_2023}. For example, If we
want to make our own snake game, we could trivially ask a model to
program the snake game.

\subsection{Copyright}

% - Go down into the history of copyright, what its
% definition is, and how it has been interpreted in the past. Really tie
% it in for the reader so it's fully clear how my argument builds upon
% this foundational information.
%   - The book by Stokes should have a lot of relevant information that
%     I can summarize and cite in this section.

% - Usage of AI tools in software engineering. Background information about
% how artificial intelligence tools have broken into the industry today
% and the prevalence of their usage. I can cite the paper:
%   - "Recent research [15] has demonstrated that large PLG [Programming
%     Language Generation] models such as GPT-J [89] can accomplish
%     computer science assignments for students without triggering MOSS
%     [76], a widely used academic plagiarism detection tool." (page 2,
%     1st paragraph, academic paper)

% \\
%thesis






\section{AI and Code Generation}

% - It's already hard to fit AI-generated anything into a box right now.
%     - Thomas's newspaper talks about how implementing regulation against
%     AI-made works has failed in the UK.

% - The lack of rules has allowed "rule breakage" of honest, well-intentioned
% standards already placed within the community, such as how AI companies
% are ignoring robots.txt.
%   - This paragraph is probably going to be a bit longer, or
%     I might split it up into two—one where I explain what a
%     robots.txt is, and the second going into the dishonest behavior.
%     https://mjtsai.com/blog/2024/06/24/ai-companies-ignoring-robots-txt/

% - Code is being written today that uses the output of these thinking
% machines—code that's going to be integrated into existing
% codebases. Any ruling in the future is going to impact code that was
% written today. It's a very murky area to be in as a company right now.

% - Additionally, as a company, it can be a liability for your own code
% through the usage of these tools because there is no guarantee that they
% aren't harvesting your code as they write it.
%     More of a cybersecurity angle.





% TODO: maybe change this name?
\section{Tracking}

% This part of my research paper is going to delve deep into the ideas and
% presentations made in the research paper about back-tracking the results of
% artificial intelligence to their source. Using this evidence, I plan to make
% a logical argument that in the future, with technology like this advancing,
% if we can find the source of any source code, and if it's similar enough,
% it makes a valid enough case for a copyright violation. Using this, I'll
% argue that for those who are concerned about the legality of their code,
% it's better to avoid using tools that spit out code, as the risk can be
% too high.

\section{Discussion and Summary}

% Summarize my point with an emphasis on my thesis.

% use section* for acknowledgment
\section*{Acknowledgment}
The authors would like to thank...



\printbibliography

\end{document}
