%% Lab Report for EEET2493_labreport_template.tex
%% V1.0
%% 2019/01/16
% 
%% This is the template for a Lab report following an IEEE paper. Modified by Francisco Tovar after Michael Sheel original document.


%% This is a skeleton file demonstrating the use of IEEEtran.cls
%% (requires IEEEtran.cls version 1.8b or later) with an IEEE
%% journal paper.
%%
%% Support sites:
%% http://www.michaelshell.org/tex/ieeetran/
%% http://www.ctan.org/pkg/ieeetran
%% and
%% http://www.ieee.org/

%%*************************************************************************
%% Legal Notice:
%% This code is offered as-is without any warranty either expressed or
%% implied; without even the implied warranty of MERCHANTABILITY or
%% FITNESS FOR A PARTICULAR PURPOSE! 
%% User assumes all risk.
%% In no event shall the IEEE or any contributor to this code be liable for
%% any damages or losses, including, but not limited to, incidental,
%% consequential, or any other damages, resulting from the use or misuse
%% of any information contained here.
%%
%% All comments are the opinions of their respective authors and are not
%% necessarily endorsed by the IEEE.
%%
%% This work is distributed under the LaTeX Project Public License (LPPL)
%% ( http://www.latex-project.org/ ) version 1.3, and may be freely used,
%% distributed and modified. A copy of the LPPL, version 1.3, is included
%% in the base LaTeX documentation of all distributions of LaTeX released
%% 2003/12/01 or later.
%% Retain all contribution notices and credits.
%% ** Modified files should be clearly indicated as such, including  **
%% ** renaming them and changing author support contact information. **
%%*************************************************************************

\documentclass[journal]{IEEEtran}

% *** CITATION PACKAGES ***
\usepackage[style=ieee]{biblatex} 
\bibliography{main.bib}    %your file created using JabRef

% *** MATH PACKAGES ***
\usepackage{amsmath}

% *** PDF, URL AND HYPERLINK PACKAGES ***
\usepackage{url}
% correct bad hyphenation here
\hyphenation{op-tical net-works semi-conduc-tor}
\usepackage{graphicx}  %needed to include png, eps figures
\usepackage{float}  % used to fix location of images i.e.\begin{figure}[H]

\begin{document}

% paper title
\title{Analyzing the Effectiveness of Copyright Law on AI Output in Software
Engineering and Future Developments}

% author names 
\author{Surendra Jammishetti, sjammish@ucsc.edu}% <-this % stops a space
        
% The report headers
\markboth{Writing 2 Final Argumentative Essay, February 2025}%do not delete next lines
{Shell \MakeLowercase{\textit{et al.}}: Bare Demo of IEEEtran.cls for IEEE Journals}

% make the title area
\maketitle

% As a general rule, do not put math, special symbols or citations
% in the abstract or keywords.
\begin{abstract}
  The increasing usage of code-generation tools by the developer community
  has not considered the legal implications of using AI-generated code. With
  model training data known to be publicly sourced, the chances of
  generated code being tied to licensed code are large. New research shows that
  origin analysis can be performed on the generated code to find its author/host
  repository which could make way for legal proceedings.  

  
\end{abstract}

\begin{IEEEkeywords}
AI, Software Engineering, Copyright
\end{IEEEkeywords}

\section{Introduction}
% want to talk about the problem space
% rise of artificial intelligence
% tools like github copilot, cursor AI editor
% widespread usage by college students,
% talk about github marketing code efficiency 
% just state thesis
%
% 

\IEEEPARstart{T}{he} emergence of AI tools has, no doubt, changed many
aspects of how work is approached in various professions,
but the greatest impact has been on software engineering.
New tools like Github Copilot, an AI-assisted code-generator
which is usable in many modern code editors today boasts
a 2x speedup of feature completions and a large boost in
productivity for its users \cite{github}. However accurate those claims
maybe, there's no doubt that these tools are being used widely
by the developer population, and they aren't going away any
time soon.
However, the code generated by these tools must come from
somewhere. Many models source their training data from the
public internet, but not everything in public access allows for
commercial/derivative use. Such models ignore this step,
jumping straight towards using this data without any regard
for copyright protections in place. While copyright has been
slow in the past, there is no doubt the legal systems around
the world will regulate and form precedents for these utilities;
Inevitably placing their users at risk.
% WARN: not sure this is the cleanest way to segue into my theses 

\textbf {
  For those concerned with the legality of the software they produce, the usage of
AI-generated code should be avoided as soon as possible, as they could become succeptible to
copyright law in the near future.
}

\section{Background}

% To help understand the context that this paper resides in, below are
% lay-person explanations of AI training, Software Licensure, AI tools, and
% more.


To help understand the context this paper resides in,
below are lay-person explanations of AI tools / training, Software
Licensure, and more.

\subsection{Data Acquisition}

The way AI development labs gather data for their models is
mainly through web scrapers. Web scrapers are programs that access a
immense number of websites, downloading and orgranizing
the all the data they find \cite{Miquido}. The gathered data
is then used to train whichever model they are interested in. 

The modern way for a website to prevent a webscraper from
downloading its contents is a file called a `robots.txt`,
that any honest web scraper would look for first, and obey
the rules inside \cite{robots}. The problem with this "trust me" approach
is that a dishonest, or even badly programmed web scraper
can come along and look past this file and do whatever they please\cite{robots}.



\subsection{Software Licenses}

% -An aside / short paragraph about the overlap between copyright and code.
% Talk about Google vs. Oracle and how modern software practices software
% through the idea of licenses. It's really important that this point
% gets understood by the reader, so add examples / pictures from GitHub?
%   - The article by Harvard Law Review going over the Google v. Oracle
%     case should give enough information to summarize the technical details
%     for the audience so they have a basic understanding. I can cite one
%     of the most widely used licenses and show how they apply to the code
%     in repositories, i.e., the MIT license.


While it may come as a surprise to the average person,
code nowadays has some kind of protection against
copying / un-permitted usage. The mechanism for this
is called a "Software License"\cite{Blackduck}.
The code that these licenses protect is called "source code".
There exist many kinds of permits, mainly meant for
open use by the public, but there are two that are crucial to understand.

\begin{IEEEitemize}
\item Copy Left 
\\
  A "Copy Left" license allows the general public to view and do whatever they
  would like with the source code, but they enforce that any derivative must
  also use the same license. In effect, this is a strictly anti-commercial license
  as it is quite hard for a business to use Copy Left Licensed software as
  anything they produce using that code must have the same
  Copy Left license, and as a result, live in the public domain.
  \cite{Blackduck}

\item Proprietary 
\\
A "Proprietary" license is the most restrictive, as it prevents any viewer/user from
copying, modifying, redistributing , etc \cite{Blackduck}. They are the defacto type of license for any
commercial code, as they protect code the best, and are legally viable \cite{harvard}. 
    
\end{IEEEitemize}



\subsection{AI-Powered Tools}

The most popular AI tool for developers is Github Copilot, which
helps developers as they write code. Its most important capability, in
the context of this work, is its power to automatically generate code when
you ask. The same capability exists for Claude, OpenAI's models, and others,
where they can take a description of what the
user wants, and can write code to perform that task \cite{s_2023}. For example, If we
want to make our own snake game, we could trivially ask a model to
program the snake game.

% - Usage of AI tools in software engineering. Background information about
% how artificial intelligence tools have broken into the industry today
% and the prevalence of their usage. I can cite the paper:
%   - "Recent research [15] has demonstrated that large PLG [Programming
%     Language Generation] models such as GPT-J [89] can accomplish
%     computer science assignments for students without triggering MOSS
%     [76], a widely used academic plagiarism detection tool." (page 2,
%     1st paragraph, academic paper)

\subsection{Copyright}

% Copyright is a mechanism used to protect created works from direct copying
% but not a sufficiently derivative use or one that gives adequate credit to
% the original author \cite[p.~55]{stokes2021}. There has been one landmark
% case regarding copyright and programming, Google INC v. Oracle America, Inc,
% which regards the copying of generic and widely known API code by Google
% from Oracle \cite{harvard}. API in short means that its code that acts as a
% abstracted layer, such that the user doesnt have to think about whats going
% on underneath. Think of it like me making my own laptop, but still keeping
% around the QWERTY keyboard so the users of my laptop dont have to learn a
% whole new keyboard layout. The ruling stated that its alright for google to
% copy this code because they were doing it in the interest of the general
% public and thought it wasnt possible to copyright something so generic,
% leading to google winning the case.


Copyright is a mechanism used to protect created works from direct copying
but not a sufficiently derivative use or one that gives adequate credit to
the original author \cite[p.~55]{stokes2021}. There has been one landmark
case regarding copyright and programming, Google INC v. Oracle America, Inc.,
which regards the copying of generic and widely known API code by Google
from Oracle \cite{harvard}. API in short means that its code acts as an
abstracted layer, such that the user doesn't have to think about what's going
on underneath. Think of it like me making a new laptop, but still keeping
around the QWERTY keyboard so the users of my laptop don't have to learn a
whole new keyboard layout. The ruling stated that it was all right for Google to
copy this code because they were doing it in the interest of the general
public and thought it wasn't possible to copyright something so generic,
leading to Google winning the case.


% - Go down into the history of copyright, what its
% definition is, and how it has been interpreted in the past. Really tie
% it in for the reader so it's fully clear how my argument builds upon
% this foundational information.
%   - The book by Stokes should have a lot of relevant information that
%     I can summarize and cite in this section.

%thesis

\section{AI and Code Generation}

% AI has been getting better and better at generating
% code \cite{codesignal}, due to the vasty increasing training data
% these models are harvesting from the web \cite{lacour_2024}.
% Especially with many popular code repositories being open online,
% its fairly straightforward to assume that they are being used for training.
% While its impossible to know exactly whats being used as training data and
% what isnt by these large companies, they arent making it any easier with their
% lack of clarity and reassurance \cite{willison_2023}. Therefore, it can be
% reasonably assumed that licensed code that is public purview is also
% being used as training data.


% The issue then arises with the licensure of the code ingested by the model.
% Whether its under a "Proprietary" license or a "Copy Left" license, the
% end user could be in trouble.

% A Proprietary license would bar any derivative usage, putting the
% user in deep trouble if found out, whether they're trying to publicize
% their code or not.

% A Copy Left, while less troublesome, an entity trying to privatize their code
% would fail to uphold the conditions of the license, which demands that any derivative
% works also be in the public domain.

% We havent even gotten to the fact that many licenses require attribution to the
% original author, which would be completely lost in this process of ai ingestion
% and generation.


AI has been getting better and better at generating
code \cite{codesignal}, due to the increasing training data
the AI companies are harvesting from the web \cite{lacour_2024}.
Especially with many popular code repositories being open source, meaning anyone
online can view them, 
it can be assumed that they are being used for training.
While it's impossible to know exactly what's being used as training data and
what isn't by these large companies, they aren't making it any easier with their
lack of clarity and reassurance \cite{willison_2023}. Therefore, it can be
reasonably assumed that licensed code that is on the public web is also
being used as training data.


The issue then arises with the licensure of the code ingested by the model.
Whether it's under a "Proprietary" license or a "Copy Left" license, the
end user could be in trouble.

A Proprietary license would bar any derivative usage, putting the
user in deep trouble if found out, whether they're trying to publicize
their code or not.

A Copy Left, while less troublesome, would mean that an entity trying to privatize their code
would fail to uphold the conditions of the license, which demands that any derivative
works also be in the public domain.

We haven't even gotten to the fact that many licenses require attribution to the
original author, which would be completely lost in this process of AI ingestion
and generation.

% - It's already hard to fit AI-generated anything into a box right now.
%     - Thomas's newspaper talks about how implementing regulation against
%     AI-made works has failed in the UK.

% - The lack of rules has allowed "rule breakage" of honest, well-intentioned
% standards already placed within the community, such as how AI companies
% are ignoring robots.txt.
%   - This paragraph is probably going to be a bit longer, or
%     I might split it up into two—one where I explain what a
%     robots.txt is, and the second going into the dishonest behavior.
%     https://mjtsai.com/blog/2024/06/24/ai-companies-ignoring-robots-txt/

% - Code is being written today that uses the output of these thinking
% machines—code that's going to be integrated into existing
% codebases. Any ruling in the future is going to impact code that was
% written today. It's a very murky area to be in as a company right now.

% - Additionally, as a company, it can be a liability for your own code
% through the usage of these tools because there is no guarantee that they
% aren't harvesting your code as they write it.
%     More of a cybersecurity angle.





% TODO: maybe change this name?
\section{Detecting the Origins of Generated Code}

% While it may seem impossible, If there was a way to figure out where
% the code generated by an AI came from, any users of the generated code
% could be in huge trouble depending on the licensure of the source code.
% Such technology doesnt exist today but is tending towards that direction,
% as seen in this study: \cite{ma2024}. They dive deep into the possibility
% of detecting whether or not code generated by a model can be traced
% back / verified to be in some dataset.

% Obviously this hasn't been attempted on mainstream models, but
% with their findings being so fruitful, its only a matter of time
% before new research builds off this foundation to see if
% generated code has its roots on public code. Logically then,
% source code authors could audit whether or not others have indirectly used their
% source code, via this AI ingestion and generation proxy. Now a solid case
% can be built up by authors with strict licenses, that this usage is
% in violation of their licenses and the derivate users now face
% legal trouble.

While it may seem impossible, If there was a way to figure out where
the code generated by an AI came from, any users of the generated code
could be in huge trouble depending on the licensure of the source code.
Such technology doesn't exist today but is tending toward that direction,
as seen in this study: \cite{ma2024}. They dive deep into the possibility
of detecting whether or not code generated by a model can be traced
back / verified to be in some dataset. They conclude that its
possible for some opensource models who are trained on specific datasets
but more is certainly possible \cite{ma2024}.

This hasn't been attempted on mainstream models, but
with their findings being so fruitful, it's only a matter of time
before new research builds off this foundation to see if
generated code has its roots in publicly availible code. Logically then,
source code authors could audit whether or not others have indirectly used their
source code, via this AI ingestion and generation proxy. Now a solid case
can be built up by authors with strict licenses, that this usage is
in violation of their licenses and the derivate users now face
legal trouble.

\section{Forward Compliance}

% While origin analysis of AI-generated code is not yet developed, preventative
% action must be taken now by those interested in upholding the legality of their
% software. As this technology develops, the risk of legal trouble only increases.

% Forward compliance simply means to disassociate with AI-generation tools, and
% return to a "traditional" style of programming. This is doubley-so for corporate
% projects which would face even higher repurcussions by any future lawsuits.

% While some may argue that the process of AI-generating code is transformative by
% defenition, but currently there exists no precedent, so claming as such is meaningless.
% When dealing with Copyright, its best to live on the side of pessimism, so
% treating AI-generation as akin to copying allows users of these tools
% to be protected in case future precedent is pessimistic as well. 

% Others can argue that the copying of code has precedent, and is allowed according
% to Google INC v. Oracle America Inc., but a deeper inspection of the case details must
% be had. The copying of the API code was only allowed because its value lay
% in the familiarity the general public already had with it. Additionally this API code
% lacked any of the deeper functionality which could be copyrighted and protected, its sematics
% ruled to be too abstract \cite{harvard}.

% This reasoning cannot be used for all code however,
% as the AI-tools can generate anything from this API-like code, to code that performs
% real work, with no care for the licensure of either. If an author were to form a case
% claiming that their proprietary code was "copied" and now exists with an unauthorized
% holder, using the afforementioned origin anazlyzer to show the means of "copying,
% the results of this case would turn out very different.


While the origin analysis of AI-generated code is not yet developed, preventative
action must be taken now by those interested in upholding the legality of their
software. As this technology develops, the risk of legal trouble only increases.

Forward compliance would simply require one to disassociate with AI-generation tools and
return to a "traditional" style of programming. This is doubly-so for corporate
projects which would face even higher repercussions by any future lawsuits.

While some may argue that the process of AI-generating code is transformative by
definition, currently there exists no precedent, so claiming as such is meaningless.
When dealing with Copyright, it's best to live on the side of pessimism, so
treating AI generation as akin to copying allows users of these tools
to be protected in case future precedent is pessimistic as well. 

Others can argue that the copying of code has precedent, and is allowed according
to Google INC v. Oracle America Inc., but a deeper inspection of the case details must
be had. The copying of the API code was only allowed because its value lay
in the familiarity the general public already had with it. Additionally, this API code
lacked any of the deeper functionality that could be copyrighted and protected, its semantics
ruled to be too abstract \cite{harvard}.

This reasoning cannot be used for all code, however,
as the AI tools can generate anything from this API-like code, to code that performs
real work, with no care for the licensure of either. If an author were to form a case
claiming that their proprietary code was "copied" and now exists with an unauthorized
holder, using the aforementioned origin analysis technique to show the means of "copying,
the results of this case would turn out very different.

% This part of my research paper is going to delve deep into the ideas and
% presentations made in the research paper about back-tracking the results of
% artificial intelligence to their source. Using this evidence, I plan to make
% a logical argument that in the future, with technology like this advancing,
% if we can find the source of any source code, and if it's similar enough,
% it makes a valid enough case for a copyright violation. Using this, I'll
% argue that for those who are concerned about the legality of their code,
% it's better to avoid using tools that spit out code, as the risk can be
% too high.

\section{Discussion and Summary}

% There exists a real legal threat, maybe not now but certainly in the near future,
% for those using AI-generated code within their produts, doubly-so if its commercial
% in nature. As of right now there is no way to tell if the code generated comes
% from a open-source repository or one thats heavily licensed. With the research
% on origin analyzers only developing, its very reasonable for AI-generated code
% to be back tracked to its source, leading users to fall under jurisdiction of
% this source code. For those concerned with the legality of their code, its
% best to stray away from these tools. 


There exists a real legal threat, maybe not now but certainly in the near future
for those using AI-generated code within their products, doubly so if it is commercial
in nature. As of right now, there is no way to tell if the code generated comes
from an open-source repository or one that is heavily licensed. With the research
of origin analyzers only developing, it's very reasonable for AI-generated code
to be backtracked to its source, leading users to fall under the jurisdiction of
this source code. For those concerned with the legality of their code, it's
best to stray away from these tools. 


% Summarize my point with an emphasis on my thesis.

% use section* for acknowledgment
\section*{Acknowledgments and Notes}
I'd like to thank Dr.Kirsch and my peer reviewers for helping this essay come together. 
One thing I'd like to note are that the text formatter im using for this template does
not allow for citations to be bolded, so note that any references to \cite{stokes2021},
\cite{harvard} and \cite{ma2024} are my peer-reviewed sources, sorry for the trouble.
My paper and citations are in the IEEE style.



\printbibliography

\end{document}
