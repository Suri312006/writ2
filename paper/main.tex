%% Lab Report for EEET2493_labreport_template.tex
%% V1.0
%% 2019/01/16
%% This is the template for a Lab report following an IEEE paper. Modified by Francisco Tovar after Michael Sheel original document.


%% This is a skeleton file demonstrating the use of IEEEtran.cls
%% (requires IEEEtran.cls version 1.8b or later) with an IEEE
%% journal paper.
%%
%% Support sites:
%% http://www.michaelshell.org/tex/ieeetran/
%% http://www.ctan.org/pkg/ieeetran
%% and
%% http://www.ieee.org/

%%*************************************************************************
%% Legal Notice:
%% This code is offered as-is without any warranty either expressed or
%% implied; without even the implied warranty of MERCHANTABILITY or
%% FITNESS FOR A PARTICULAR PURPOSE! 
%% User assumes all risk.
%% In no event shall the IEEE or any contributor to this code be liable for
%% any damages or losses, including, but not limited to, incidental,
%% consequential, or any other damages, resulting from the use or misuse
%% of any information contained here.
%%
%% All comments are the opinions of their respective authors and are not
%% necessarily endorsed by the IEEE.
%%
%% This work is distributed under the LaTeX Project Public License (LPPL)
%% ( http://www.latex-project.org/ ) version 1.3, and may be freely used,
%% distributed and modified. A copy of the LPPL, version 1.3, is included
%% in the base LaTeX documentation of all distributions of LaTeX released
%% 2003/12/01 or later.
%% Retain all contribution notices and credits.
%% ** Modified files should be clearly indicated as such, including  **
%% ** renaming them and changing author support contact information. **
%%*************************************************************************

\documentclass[journal]{IEEEtran}

% *** CITATION PACKAGES ***
\usepackage[style=ieee]{biblatex} 
\bibliography{main.bib}    %your file created using JabRef

% *** MATH PACKAGES ***
\usepackage{amsmath}

% *** PDF, URL AND HYPERLINK PACKAGES ***
\usepackage{url}
% correct bad hyphenation here
\hyphenation{op-tical net-works semi-conduc-tor}
\usepackage{graphicx}  %needed to include png, eps figures
\usepackage{float}  % used to fix location of images i.e.\begin{figure}[H]

\begin{document}

% paper title
\title{Analyzing the Effectiveness of Copyright Law on AI Output in Software
Engineering and Future Developments}

% author names 
\author{Surendra Jammishetti, sjammish@ucsc.edu}% <-this % stops a space
        
% The report headers
\markboth{Writing 2 Final Argumentative Essay, February 2025}%do not delete next lines
{Shell \MakeLowercase{\textit{et al.}}: Bare Demo of IEEEtran.cls for IEEE Journals}

% make the title area
\maketitle

% As a general rule, do not put math, special symbols or citations
% in the abstract or keywords.
\begin{abstract}
TODO: need to do this later
\end{abstract}

\begin{IEEEkeywords}
AI, Software Engineering, Copyright
\end{IEEEkeywords}

\section{Introduction}
% want to talk about the problem space
% rise of artificial intelligence
% tools like github copilot, cursor AI editor
% widespread usage by college students,
% talk about github marketing code efficiency 
% just state thesis
%
% 

\IEEEPARstart{W}{rite} why is important to do this experiment, what background is needed, what technology has been used in this session, you can also talk briefly about what other technology exist but was not used here.
Then explain briefly how the experiment was conducted, what measurements were taken, what technology is used (acquisition system, sensors, software), if calculations were done, what calculations were done, what decisions were made, and what the final result was (explained in a concise way with words).
Writing “good” reports requires much thought, organization and editing but the rewards are
great. Those students who can master good technical writing skills will find greater success and
opportunity as professionals in industry.

% \\
%thesis
For those concerned with the legality of the software they produce, the usage of
AI-generated code should be avoided as their trustworthiness is dubious and
they will become succeptible to copyright law in the near future.




\section{Background}

% - Go down into the history of copyright, what its
% definition is, and how it has been interpreted in the past. Really tie
% it in for the reader so it's fully clear how my argument builds upon
% this foundational information.
%   - The book by Stokes should have a lot of relevant information that
%     I can summarize and cite in this section.

% -An aside / short paragraph about the overlap between copyright and code.
% Talk about Google vs. Oracle and how modern software practices software
% through the idea of licenses. It's really important that this point
% gets understood by the reader, so add examples / pictures from GitHub?
%   - The article by Harvard Law Review going over the Google v. Oracle
%     case should give enough information to summarize the technical details
%     for the audience so they have a basic understanding. I can cite one
%     of the most widely used licenses and show how they apply to the code
%     in repositories, i.e., the MIT license.

% - Usage of AI tools in software engineering. Background information about
% how artificial intelligence tools have broken into the industry today
% and the prevalence of their usage. I can cite the paper:
%   - "Recent research [15] has demonstrated that large PLG [Programming
%     Language Generation] models such as GPT-J [89] can accomplish
%     computer science assignments for students without triggering MOSS
%     [76], a widely used academic plagiarism detection tool." (page 2,
%     1st paragraph, academic paper)

\section{AI and Code Generation}

% - It's already hard to fit AI-generated anything into a box right now.
%     - Thomas's newspaper talks about how implementing regulation against
%     AI-made works has failed in the UK.

% - The lack of rules has allowed "rule breakage" of honest, well-intentioned
% standards already placed within the community, such as how AI companies
% are ignoring robots.txt.
%   - This paragraph is probably going to be a bit longer, or
%     I might split it up into two—one where I explain what a
%     robots.txt is, and the second going into the dishonest behavior.
%     https://mjtsai.com/blog/2024/06/24/ai-companies-ignoring-robots-txt/

% - Code is being written today that uses the output of these thinking
% machines—code that's going to be integrated into existing
% codebases. Any ruling in the future is going to impact code that was
% written today. It's a very murky area to be in as a company right now.

% - Additionally, as a company, it can be a liability for your own code
% through the usage of these tools because there is no guarantee that they
% aren't harvesting your code as they write it.
%     More of a cybersecurity angle.





% TODO: maybe change this name?
\section{Tracking}

% This part of my research paper is going to delve deep into the ideas and
% presentations made in the research paper about back-tracking the results of
% artificial intelligence to their source. Using this evidence, I plan to make
% a logical argument that in the future, with technology like this advancing,
% if we can find the source of any source code, and if it's similar enough,
% it makes a valid enough case for a copyright violation. Using this, I'll
% argue that for those who are concerned about the legality of their code,
% it's better to avoid using tools that spit out code, as the risk can be
% too high.

\section{Discussion and Summary}

% Summarize my point with an emphasis on my thesis.




\appendices
\section{Hand calculations (or name your title for appendix subtitle)}
List any extra evidence such as photos of the session, that may help you support your claims.
You can include all hand calculations, extra graphs and plots, simulation results, etc. 

% use section* for acknowledgment
\section*{Acknowledgment}
The authors would like to thank...



\printbibliography

% Examples of references:  \\[0.001in]

% Example of data book:\\[0.1in]
% [2] National Operational Amplifiers Databook. Santa Clara: National Semiconductor
% Corporation, 1995 Edition, p. I-54. \\[0.1in]
% Example of textbook: \\[0.1in]
% [3]M. Young, The Technical Writer’s Handbook. Mill Valley, CA: University Science, 1989.\\[0.1in]
% Example of scientific journal paper:\\[0.1in]
% [4] J.W. Smith, L.S. Alans and D.K. Jones, “An operational amplifier approach to
% active cable modeling”, IEEE Transactions on Modeling, vol. 4, no. 2, 1996, pp.
% 128-132.\\[0.1in]
% Example of conference paper proceedings:\\[0.1in]
% [5] J.W. Smith, L.S. Alans and D.K. Jones, “Active cable models for lossy
% transmission line circuits”, in Proc. 1995 IEEE Modeling Symposium, 1996, pp.
% 1086-89.\\[0.1in]

% Example of Internet web page:\\[0.1in]
% [6] Approximate material properties in isotropic materials. Milpitas, CA: Specialty
% Engineering Associates, Inc. web site: www.ultrasonic.com, downloaded Aug. 20,
% 2001. 

% List and number all bibliographical 
% references at the end of your paper in {\bf 9 or 10 point} Times, with 10-point interline spacing. When referenced within the text, enclose the citation number in square brackets, for example [1]. \\
% Use IEEE format. Cite any external work that you used (data sheets, text books, Wikipedia articles, . . . ). If you get a formula from a Wikipedia article, you must cite the article, giving the title, the URL, and the data you accessed the article as a minimum. If you copy a figure, not only must you cite the article you copied from, but you must give explicit figure credit in the caption for the figure: This image copied from . . . . If you modify a figure or base your figure on one that has been published elsewhere, you still need to give credit in the caption: This image adapted from . . . .\\[0.1in]

% that's all folks
\end{document}
