
\documentclass[12pt]{article}

%
%Margin - 1 inch on all sides
%
\usepackage[letterpaper]{geometry}
\usepackage{times}
\geometry{top=1.0in, bottom=1.0in, left=1.0in, right=1.0in}

%
%Doublespacing
%
\usepackage{setspace}
\doublespacing

%
%Rotating tables (e.g. sideways when too long)
%
\usepackage{rotating}


%
%Fancy-header package to modify header/page numbering (insert last name)
%
\usepackage{fancyhdr}
\pagestyle{fancy}
\lhead{}
\chead{}
\rhead{Jammishetti \thepage}
\lfoot{}
\cfoot{}
\rfoot{}
\renewcommand{\headrulewidth}{0pt}
\renewcommand{\footrulewidth}{0pt}
%To make sure we actually have header 0.5in away from top edge
%12pt is one-sixth of an inch. Subtract this from 0.5in to get headsep value
\setlength\headsep{0.333in}

%
%Works cited environment
%(to start, use \begin{workscited...}, each entry preceded by \bibent)
% - from Ryan Alcock's MLA style file
%
\newcommand{\bibent}{\noindent \hangindent 40pt}
\newenvironment{workscited}{\newpage \begin{center} Works Cited \end{center}}{\newpage }


%
%Begin document
%
\begin{document}
\begin{flushleft}

	%%%%First page name, class, etc
	Surendra Jammishetti\\
	Professor Kirsh\\
	Writing 2\\
	January 07 2025\\


	%%%%Title
	\begin{center}
		Project 1
	\end{center}


	%%%%Changes paragraph indentation to 0.5in
	\setlength{\parindent}{0.5in}
	%%%%Begin body of paper here

	I'd say I'm much more of a reader than a writer, and frankly, I'm disappointed about that.
	Maybe that's because I'm burnt out after college apps / AP tests where I had to think
	about what others wanted to hear and follow a semi-strict format, instead of writing what
	I wanted to write. The most natural style of writing is my inner monologue, especially when
	I'm writing comments in my code. There'll be questions and short snips, and I like
	doing that because it feels more personal. It leaves a part of me inside of it
	rather than it just being some obscure computer program. Other than that, I think that
	my writing doesn't have a voice.

	While I'd say I was pretty good at "academic" writing back in high school,
	it never really felt like I was just writing. I desire the feeling of writing
	down my thoughts, where it feels like the reader is going to get to know me as
	a person. That's the kind of writing I enjoy. The kind of writing that doesn't follow
	conventional rules, and is rather more personable. I want my writing to be a reflection
	of me, like a hologram or a phone call.

	Now the majority of my writing consists of text messages, and coding, not sure if that
	counts or not. According to PACT, I "am" writing my code with a purpose, my audience being
	the users, within the context of a project / my interests, and yeah code is textual so I'm hoping that
	makes me a writer. Now that I think about it, writing anything does have a certain
	depth to it, which is why I'm partially disappointed in myself for not writing more. It
	feels like I'm missing out on an integral part of the human experience.

	I value writing skills in today's world, mostly because of what Genai is
	allowing us students to do. I dislike how it's being used to "complete"
	things. It has always struck me the wrong way, because I take pride in the work
	that I do, so using something else and calling it mine just feels paradoxical.
	Additionally, it all sounds the same. There's no personality. There was this
	critique I saw, where Apple is using AI to summarize text messages, and they are
	also using AI to write text messages, you can give it a summary of what you want
	it to say and it generate it. What does this mean? The future is going to
	consist of us writing bullet points and reading bullet points and there's
	going to be bland and verbose text as the middle man. It's disturbing. I want
	to take pride in the things I write, being able to say that these
	words came from my head, unassisted, these words are a snapshot of myself in time.

	I tried to start a blog a couple of months back, it's on my website (www.suri.codes if you're interested in checking it out).
	I was excited to write about some of the things I was working on and got one
	blog post successfully made but that was the end of that. As school, research, and work
	just started piling up, I lost interest, and motivation, to keep working on the blog.
	I'm hoping that this class helps me flex my muscles enough, and reignites the spark so
	I can at least get a couple more posts up on my blog.



	\newpage


	%%%%Title
	% \begin{center}
	% 	Notes
	% \end{center}


	% \setlength{\parindent}{0.5in}

	% 1. Danhof includes “Delaware, Maryland, all states north of the Potomac and Ohio rivers, Missouri, and states to its north” when referring to the northern states (11).


	% 2. For the purposes of this paper,“science” is defined as it was in nineteenthcentury agriculture: conducting experiments and engaging in research.


	% 3. Please note that any direct quotes from the nineteenth century texts are writtenin their original form, which may contain grammar mistakes according to twenty-first century grammar rules.

	%%%%Works cited
	% \begin{workscited}

	% 	\bibent
	% 	Allen, R.L. \textit{The American Farm Book; or Compend of Ameri can Agriculture; Being a Practical Treatise on Soils, Manures, Draining, Irrigation, Grasses, Grain, Roots, Fruits, Cotton, Tobacco, Sugar Cane, Rice, and Every Staple Product of the United States with the Best Methods of Planting, Cultivating, and Prep aration for Market.} New York: Saxton, 1849. Print.

	% 	\bibent
	% 	Baker, Gladys L., Wayne D. Rasmussen, Vivian Wiser, and Jane M. Porter. \textit{Century of Service: The First 100 Years of the United States Department of Agriculture.}[Federal Government], 1996. Print.

	% 	\bibent
	% 	Danhof, Clarence H. \textit{Change in Agriculture: The Northern United States, 1820-1870.} Cambridge: Harvard UP, 1969. Print.


	% \end{workscited}

\end{flushleft}
\end{document}
\}
